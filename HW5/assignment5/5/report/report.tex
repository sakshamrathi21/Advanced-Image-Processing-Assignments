\documentclass[a4paper,12pt]{article}
\usepackage{xcolor}
\usepackage{amsmath,amsfonts,amssymb}
\usepackage{geometry}
\usepackage{fancyhdr}
\usepackage{graphicx}
\usepackage{titlesec}
\usepackage{tikz}
\usepackage{booktabs}
\usepackage{array}
\usetikzlibrary{shadows}
\usepackage{tcolorbox}
\usepackage{float}
\usepackage{lipsum}
\usepackage{mdframed}
\usepackage{pagecolor}
\usepackage{mathpazo}   % Palatino font (serif)
\usepackage{microtype}  % Better typography
\usepackage{subfigure}
% \hypersetup{
%   colorlinks=true,
%   linkcolor=blue,
%   urlcolor=blue,
%   citecolor=blue
% }

% Page background color
\pagecolor{gray!10!white}

% Geometry settings
\geometry{margin=0.5in}
\pagestyle{fancy}
\fancyhf{}

% Fancy header and footer
\fancyhead[C]{\textbf{\color{blue!80}CS754 Assignment-5}}
% \fancyhead[R]{\color{blue!80}Saksham Rathi}
\fancyfoot[C]{\thepage}

% Custom Section Color and Format with Sans-serif font
\titleformat{\section}
{\sffamily\color{purple!90!black}\normalfont\Large\bfseries}
{\thesection}{1em}{}

% Custom subsection format
\titleformat{\subsection}
{\sffamily\color{cyan!80!black}\normalfont\large\bfseries}
{\thesubsection}{1em}{}

% Stylish Title with TikZ (Enhanced with gradient)
\newcommand{\cooltitle}[1]{%
  \begin{tikzpicture}
    \node[fill=blue!20,rounded corners=10pt,inner sep=12pt, drop shadow, top color=blue!50, bottom color=blue!30] (box)
    {\Huge \bfseries \color{black} #1};
  \end{tikzpicture}
}
\usepackage{float} % Add this package

\newenvironment{solution}[2][]{%
    \begin{mdframed}[linecolor=blue!70!black, linewidth=2pt, roundcorner=10pt, backgroundcolor=yellow!10!white, skipabove=12pt, skipbelow=12pt]%
        \textbf{\large #2}
        \par\noindent\rule{\textwidth}{0.4pt}
}{
    \end{mdframed}
}

% Document title
\title{\cooltitle{CS754 Assignment-5}}
\author{{\bf Saksham Rathi, Ekansh Ravi Shankar, Kshitij Vaidya}}
\date{}

\begin{document}
\maketitle
\textbf{Declaration:} The work submitted is our own, and
we have adhered to the principles of academic honesty while completing and submitting this work. We have not referred to any unauthorized sources, and we have not used generative AI tools for the work submitted here.

\section*{Question 5}

\begin{solution}{Solution}

\subsection{Part (a)}

\subsubsection{Cost Function:}
\begin{equation}
J_1(A_r) = \|A - A_r\|_F^2
\end{equation}
where \( A \in \mathbb{R}^{m \times n} \) is a known matrix, and \( A_r \) is a rank-\( r \) approximation of \( A \) with \( r < m, r < n \).

\subsubsection{Minimization Method:}  
Compute the rank-\( r \) approximation \( A_r \) using the \textbf{Singular Value Decomposition (SVD)} of \( A \). Let:
\begin{equation}
A = U \Sigma V^T
\end{equation}
Then the best rank-\( r \) approximation is:
\begin{equation}
A_r = U_r \Sigma_r V_r^T
\end{equation}
where \( \Sigma_r \) retains only the top \( r \) singular values.

\subsubsection{Application}  
\textbf{Image Compression.} Low-rank approximation is widely used to compress images while retaining the most significant information.

\subsection{Part (b)}

\subsubsection{Cost Function}
\begin{equation}
J_2(R) = \|A - RB\|_F^2
\end{equation}
where \( A, B \in \mathbb{R}^{n \times m} \), \( m > n \), and \( R \in \mathbb{R}^{n \times n} \) is an orthonormal matrix (\( R^T R = I \)).

\subsubsection{Minimization Method}  
We compute the SVD of \( AB^T \):
\begin{equation}
AB^T = U \Sigma V^T
\end{equation}
Then the optimal \( R \) is:
\begin{equation}
R = UV^T
\end{equation}

\subsubsection{Application}  
\textbf{Image Registration.} Used to align two sets of landmark points or feature descriptors under rotation/reflection.

\subsection{Part (c)}

\subsubsection{Cost Function}
\begin{equation}
J_3(A) = \|C - A\|_F^2 + \lambda \|A\|_1
\end{equation}
where \( C \in \mathbb{R}^{m \times n} \) is known and \( \|\cdot\|_1 \) denotes element-wise \( \ell_1 \)-norm.

\subsubsection{Minimization Method}  
This is a \textbf{soft-thresholding problem}, solved using the \textbf{Iterative Shrinkage-Thresholding Algorithm (ISTA)} or \textbf{FISTA} for faster convergence. A closed-form solution for element-wise shrinkage is:
\begin{equation}
A_{ij} = \text{sign}(C_{ij}) \cdot \max(|C_{ij}| - \lambda, 0)
\end{equation}

\subsubsection{Application}  
\textbf{Image Denoising.} Promotes sparsity to eliminate noise while preserving important signal components.

\subsection{Part (d)}

\subsubsection{Cost Function}
\begin{equation}
J_4(A) = \|C - A\|_F^2 + \lambda \|A\|_*
\end{equation}
where \( \|\cdot\|_* \) is the nuclear norm (sum of singular values), and \( C \in \mathbb{R}^{m \times n} \) is known.

\subsubsection{Minimization Method}  
Solved using the \textbf{Singular Value Thresholding (SVT)} algorithm. Let \( C = U \Sigma V^T \) be the SVD of \( C \). Then apply soft-thresholding to singular values:
\begin{equation}
A = U \cdot \text{diag}(\max(\sigma_i - \lambda, 0)) \cdot V^T
\end{equation}

\subsubsection{Application}  
\textbf{Image Inpainting.} Used to recover missing parts of an image by promoting low-rank structure.

\end{solution}


\end{document}
