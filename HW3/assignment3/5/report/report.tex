\documentclass[a4paper,12pt]{article}
\usepackage{xcolor}
\usepackage{amsmath,amsfonts,amssymb}
\usepackage{geometry}
\usepackage{fancyhdr}
\usepackage{graphicx}
\usepackage{titlesec}
\usepackage{tikz}
\usepackage{booktabs}
\usepackage{array}
\usetikzlibrary{shadows}
\usepackage{tcolorbox}
\usepackage{float}
\usepackage{lipsum}
\usepackage{mdframed}
\usepackage{pagecolor}
\usepackage{mathpazo}   % Palatino font (serif)
\usepackage{microtype}  % Better typography

% Page background color
\pagecolor{gray!10!white}

% Geometry settings
\geometry{margin=0.5in}
\pagestyle{fancy}
\fancyhf{}

% Fancy header and footer
\fancyhead[C]{\textbf{\color{blue!80}CS754 Assignment-1}}
% \fancyhead[R]{\color{blue!80}Saksham Rathi}
\fancyfoot[C]{\thepage}

% Custom Section Color and Format with Sans-serif font
\titleformat{\section}
{\sffamily\color{purple!90!black}\normalfont\Large\bfseries}
{\thesection}{1em}{}

% Custom subsection format
\titleformat{\subsection}
{\sffamily\color{cyan!80!black}\normalfont\large\bfseries}
{\thesubsection}{1em}{}

% Stylish Title with TikZ (Enhanced with gradient)
\newcommand{\cooltitle}[1]{%
  \begin{tikzpicture}
    \node[fill=blue!20,rounded corners=10pt,inner sep=12pt, drop shadow, top color=blue!50, bottom color=blue!30] (box)
    {\Huge \bfseries \color{black} #1};
  \end{tikzpicture}
}
\usepackage{float} % Add this package

\newenvironment{solution}[2][]{%
    \begin{mdframed}[linecolor=blue!70!black, linewidth=2pt, roundcorner=10pt, backgroundcolor=yellow!10!white, skipabove=12pt, skipbelow=12pt]%
        \textbf{\large #2}
        \par\noindent\rule{\textwidth}{0.4pt}
}{
    \end{mdframed}
}

% Document title
\title{\cooltitle{CS754 Assignment-3}}
\author{{\bf Saksham Rathi, Ekansh Ravi Shankar, Kshitij Vaidya}}
\date{}

\begin{document}
\maketitle
\textbf{Declaration:} The work submitted is our own, and
we have adhered to the principles of academic honesty while completing and submitting this work. We have not referred to any unauthorized sources, and we have not used generative AI tools for the work submitted here.

\section*{Question 5}

\begin{solution}{Solution}

\section{Relationship Between Shifted Particle Images}

Consider two observed particle images Q1 and Q2 corresponding to a 3D density map, each in different 3D orientations and 2D shifts. Let Q1 be obtained by translating a zero-shift particle image P1 by $(\delta x_1, \delta y_1)$. Let Q2 be obtained by translating a zero-shift particle image P2 by $(\delta x_2, \delta y_2)$. The common line for the particle images P1, P2 passes through the origins of their respective coordinate systems at angles $\theta_1$ and $\theta_2$ with respect to their respective X axes.

\subsection{Mathematical Derivation}
\begin{enumerate}

\item In real space, the relationship between the shifted and unshifted images is:

   \begin{align}
   Q_1(x,y) &= P_1(x-\delta x_1, y-\delta y_1) \\
   Q_2(x,y) &= P_2(x-\delta x_2, y-\delta y_2)
   \end{align}

\item We can take the Fourier Transform of the System to obtain:

   \begin{align}
   \mathcal{F}[Q_1](s,t) &= e^{-2\pi i(s\delta x_1 + t\delta y_1)} \cdot \mathcal{F}[P_1](s,t) \\
   \mathcal{F}[Q_2](s,t) &= e^{-2\pi i(s\delta x_2 + t\delta y_2)} \cdot \mathcal{F}[P_2](s,t)
   \end{align}

\item Along the common line, the Fourier space coordinates can be written as:

   \begin{align}
   \text{For Q1:} \quad (s,t) &= (r\cos\theta_1, r\sin\theta_1) \\
   \text{For Q2:} \quad (s,t) &= (r\cos\theta_2, r\sin\theta_2)
   \end{align}

   where $r$ is the distance from the origin in Fourier space.

\item The Fourier slice theorem implies that along this common line:

   \begin{equation}
   \mathcal{F}[P_1](r\cos\theta_1, r\sin\theta_1) = \mathcal{F}[P_2](r\cos\theta_2, r\sin\theta_2)
   \end{equation}

\item Therefore, we can write:

   \begin{equation}
   \frac{\mathcal{F}[Q_1](r\cos\theta_1, r\sin\theta_1)}{\mathcal{F}[Q_2](r\cos\theta_2, r\sin\theta_2)} = \frac{e^{-2\pi i(r\cos\theta_1\delta x_1 + r\sin\theta_1\delta y_1)}}{e^{-2\pi i(r\cos\theta_2\delta x_2 + r\sin\theta_2\delta y_2)}}
   \end{equation}

\item Simplifying:

   \begin{equation}
   \frac{\mathcal{F}[Q_1](r\cos\theta_1, r\sin\theta_1)}{\mathcal{F}[Q_2](r\cos\theta_2, r\sin\theta_2)} = e^{-2\pi i r[(\delta x_1\cos\theta_1 + \delta y_1\sin\theta_1) - (\delta x_2\cos\theta_2 + \delta y_2\sin\theta_2)]}
   \end{equation}

\item The phase difference between corresponding points on the common line is:

   \begin{equation}
   \Delta\phi(r) = -2\pi r[(\delta x_1\cos\theta_1 + \delta y_1\sin\theta_1) - (\delta x_2\cos\theta_2 + \delta y_2\sin\theta_2)]
   \end{equation}

\item This is a linear function of $r$ with slope:

   \begin{equation}
   m = -2\pi[(\delta x_1\cos\theta_1 + \delta y_1\sin\theta_1) - (\delta x_2\cos\theta_2 + \delta y_2\sin\theta_2)]
   \end{equation}

\end{enumerate}


\subsection{Determining Shifts}

To determine $\delta x_1, \delta y_1, \delta x_2, \delta y_2$:
\begin{enumerate}
\item Identify the common line and determine $\theta_1$ and $\theta_2$.
\item Compute the Fourier transforms of Q1 and Q2.
\item Calculate the phase difference $\Delta\phi(r)$ along the common line.
\item Measure the slope $m$ of this phase difference.
\end{enumerate}

\noindent This gives one equation with four unknowns. To solve:
\begin{enumerate}
\item Set one image (e.g., Q1) as reference: $(\delta x_1, \delta y_1) = (0, 0)$
\item Use multiple common lines between pairs of images for additional equations
\end{enumerate}


\subsection{Extension to N Images}

For N images, we have 2N unknowns $(\delta x_i, \delta y_i)$ for $i = 1$ to $N$.
\begin{enumerate}
\item For each pair of images (i,j), identify their common line and angles $\theta_i$ and $\theta_j$.
\item Compute the phase difference slope along each common line.
\item Set up a system of equations. For each pair (i,j):

   \begin{equation}
   m_{ij} = -2\pi[(\delta x_i\cos\theta_i + \delta y_i\sin\theta_i) - (\delta x_j\cos\theta_j + \delta y_j\sin\theta_j)]
   \end{equation}

\item We can form up to $\binom{N}{2}$ equations from all possible image pairs.
\item Solve the system using least squares optimization when we have more equations than unknowns (typically when $N > 4$).
\end{enumerate}
\subsection{Number of Knowns and Unknowns}

\begin{enumerate}
\item Unknowns: 2N (the shifts $\delta x_i, \delta y_i$ for $i = 1$ to $N$)
\item Knowns: Up to $\binom{N}{2}$ equations from common line pairs
\item Additional constraint: Can set one image as reference, reducing unknowns to $2(N-1)$
\end{enumerate}

\end{solution}


\end{document}