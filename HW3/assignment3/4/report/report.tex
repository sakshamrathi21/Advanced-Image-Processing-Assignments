\documentclass[a4paper,12pt]{article}
\usepackage{xcolor}
\usepackage{amsmath,amsfonts,amssymb}
\usepackage{geometry}
\usepackage{fancyhdr}
\usepackage{graphicx}
\usepackage{titlesec}
\usepackage{tikz}
\usepackage{booktabs}
\usepackage{array}
\usetikzlibrary{shadows}
\usepackage{tcolorbox}
\usepackage{float}
\usepackage{lipsum}
\usepackage{mdframed}
\usepackage{pagecolor}
\usepackage{mathpazo}   % Palatino font (serif)
\usepackage{microtype}  % Better typography

% Page background color
\pagecolor{gray!10!white}

% Geometry settings
\geometry{margin=0.5in}
\pagestyle{fancy}
\fancyhf{}

% Fancy header and footer
\fancyhead[C]{\textbf{\color{blue!80}CS754 Assignment-3}}
% \fancyhead[R]{\color{blue!80}Saksham Rathi}
\fancyfoot[C]{\thepage}

% Custom Section Color and Format with Sans-serif font
\titleformat{\section}
{\sffamily\color{purple!90!black}\normalfont\Large\bfseries}
{\thesection}{1em}{}

% Custom subsection format
\titleformat{\subsection}
{\sffamily\color{cyan!80!black}\normalfont\large\bfseries}
{\thesubsection}{1em}{}

% Stylish Title with TikZ (Enhanced with gradient)
\newcommand{\cooltitle}[1]{%
  \begin{tikzpicture}
    \node[fill=blue!20,rounded corners=10pt,inner sep=12pt, drop shadow, top color=blue!50, bottom color=blue!30] (box)
    {\Huge \bfseries \color{black} #1};
  \end{tikzpicture}
}
\usepackage{float} % Add this package

\newenvironment{solution}[2][]{%
    \begin{mdframed}[linecolor=blue!70!black, linewidth=2pt, roundcorner=10pt, backgroundcolor=yellow!10!white, skipabove=12pt, skipbelow=12pt]%
        \textbf{\large #2}
        \par\noindent\rule{\textwidth}{0.4pt}
}{
    \end{mdframed}
}

% Document title
\title{\cooltitle{CS754 Assignment-3}}
\author{{\bf Saksham Rathi, Ekansh Ravi Shankar, Kshitij Vaidya}}
\date{}

\begin{document}
\maketitle
\textbf{Declaration:} The work submitted is our own, and
we have adhered to the principles of academic honesty while completing and submitting this work. We have not referred to any unauthorized sources, and we have not used generative AI tools for the work submitted here.

\section*{Question 4}

\begin{solution}{Solution}
 The Radon Transform of a 2D image $f(x, y)$ at an angle $\theta$ is defined as:

 \begin{equation}
  \mathcal{R}_\theta f(\rho) = \int_{-\infty}^{\infty}f(x, y)\delta(\rho - (x cos\theta + y sin\theta)) dx dy
  \label{eq1}
 \end{equation}

 Since, $g(x, y)$ is another 2D image which is a version of $f(x, y)$ shifted by $(x_0, y_0)$, we can write it as follows:

 \[g(x, y) = f(x - x_0, y - y_0)\]

 The Radon transform of $g(x, y)$ at an angle $\theta$ is:

 \begin{equation}
  \mathcal{R}_\theta g(\rho) = \int_{-\infty}^{\infty}g(x, y)\delta(\rho - (x cos\theta + y sin\theta)) dx dy 
 \end{equation}

 \begin{equation}
  \mathcal{R}_\theta g(\rho)  = \int_{-\infty}^{\infty}f(x - x_0, y - y_0)\delta(\rho - (x cos\theta + y sin\theta)) dx dy
 \end{equation}

 Let $x' = x - x_0$ and $y' = y - y_0$ 

 \begin{equation}
  \mathcal{R}_\theta g(\rho)  = \int_{-\infty}^{\infty}f(x', y')\delta(\rho - ((x' + x_0) cos\theta + (y' + y_0) sin\theta)) dx dy
 \end{equation}

 \begin{equation}
  \mathcal{R}_\theta g(\rho)  = \int_{-\infty}^{\infty}f(x', y')\delta((\rho - x_0 cos\theta - y_0 sin\theta)- (x' cos\theta + y' sin\theta)) dx dy
 \end{equation}

 One can easily notice the similarity of this equation with Eq:\ref{eq1} with a modified $\rho' =  (\rho - x_0 cos\theta - y_0 sin\theta)$.

 Therefore, 
 \begin{equation}
  \mathcal{R}_\theta g(\rho)  = \mathcal{R}_\theta f(\rho') = \mathcal{R}_\theta f(\rho - x_0 cos\theta - y_0 sin\theta) = \mathcal{R}_\theta f(\rho - (x_0, y_0)\cdot(cos\theta, sin\theta))
 \end{equation}






\end{solution}


\end{document}
