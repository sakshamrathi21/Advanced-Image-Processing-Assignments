\documentclass[a4paper,12pt]{article}
\usepackage{xcolor}
\usepackage{amsmath,amsfonts,amssymb}
\usepackage{geometry}
\usepackage{fancyhdr}
\usepackage{graphicx}
\usepackage{titlesec}
\usepackage{tikz}
\usepackage{booktabs}
\usepackage{array}
\usetikzlibrary{shadows}
\usepackage{tcolorbox}
\usepackage{float}
\usepackage{lipsum}
\usepackage{mdframed}
\usepackage{pagecolor}
\usepackage{mathpazo}   % Palatino font (serif)
\usepackage{microtype}  % Better typography

% Page background color
\pagecolor{gray!10!white}

% Geometry settings
\geometry{margin=0.5in}
\pagestyle{fancy}
\fancyhf{}

% Fancy header and footer
\fancyhead[C]{\textbf{\color{blue!80}CS754 Assignment-1}}
% \fancyhead[R]{\color{blue!80}Saksham Rathi}
\fancyfoot[C]{\thepage}

% Custom Section Color and Format with Sans-serif font
\titleformat{\section}
{\sffamily\color{purple!90!black}\normalfont\Large\bfseries}
{\thesection}{1em}{}

% Custom subsection format
\titleformat{\subsection}
{\sffamily\color{cyan!80!black}\normalfont\large\bfseries}
{\thesubsection}{1em}{}

% Stylish Title with TikZ (Enhanced with gradient)
\newcommand{\cooltitle}[1]{%
  \begin{tikzpicture}
    \node[fill=blue!20,rounded corners=10pt,inner sep=12pt, drop shadow, top color=blue!50, bottom color=blue!30] (box)
    {\Huge \bfseries \color{black} #1};
  \end{tikzpicture}
}
\usepackage{float} % Add this package

\newenvironment{solution}[2][]{%
    \begin{mdframed}[linecolor=blue!70!black, linewidth=2pt, roundcorner=10pt, backgroundcolor=yellow!10!white, skipabove=12pt, skipbelow=12pt]%
        \textbf{\large #2}
        \par\noindent\rule{\textwidth}{0.4pt}
}{
    \end{mdframed}
}

% Document title
\title{\cooltitle{CS754 Assignment-2}}
\author{{\bf Saksham Rathi, Ekansh Ravi Shankar, Kshitij Vaidya}}
\date{}

\begin{document}
\maketitle
\textbf{Declaration:} The work submitted is our own, and
we have adhered to the principles of academic honesty while completing and submitting this work. We have not referred to any unauthorized sources, and we have not used generative AI tools for the work submitted here.

\section*{Question 1}

\begin{solution}{Solution}
\vspace{-25pt}
\subsection*{Step 1}
We have, for any $y \in \mathbb{R}^m$, $||\Phi(x^*-x)||_{l_2} = ||\Phi x^*-y+y-\Phi x||_{l_2} \leq ||\Phi x^* - y||_{l_2} + ||y - \Phi x||_{l_2}$ by Triangle Inequality, i.e magnitude of sum of two vectors is always less than or equal to the sum of magnitudes of the vectors. \\
Also, by (7) in the paper, we want to find $\tilde{x}$ subject to $||y-\Phi x|| \leq \varepsilon$. Thus, $||\Phi x^* - y||_{l_2} + ||y - \Phi x||_{l_2} \leq \varepsilon + \varepsilon = 2\varepsilon$
Thus,
\[
||\phi(x^*-x)||_{l_2} \leq ||\phi x^* - y||_{l_2} + ||y - \phi x||_{l_2} \leq 2\varepsilon
\]

\subsection*{Step 2}

If a vector $v$ has atmost $s$ nonzero entries, we have $||v||_{l_2} \leq \sqrt{s}||v||_{l_\infty}$, because the squared $l_2$-norm is the sum of squares of at most $s$ elements, and each term is at most $||v||_{l_\infty}$. Thus, $||h_{T_j}||_{l_2} \leq s^{1/2}||h_{T_j}||_{l_\infty}$.\\
Now, $T_j$ consists of the largest $s$ entries of $h$ after removing the previous sets $T_0, \dots, T_{j-1}$. Thus, the largest element in $h_{T_j}$ is atmost the average of magnitude of all elements in $h_{T_{j-1}}$. Thus, $||h_{T_j}||_{l_\infty} \leq \frac{1}{s}||h_{T_{j-1}}||_{l_1}$. Combining the two results, we get the required inequality.
\[
||h_{T_j}||_{l_2} \leq s^{1/2}||h_{T_j}||_{l_\infty} \leq s^{-1/2}||h_{T_{j-1}}||_{l_1}
\]

\subsection*{Step 3}
The first inequality is obvious from the result above, that $||h_{T_j}||_{l_2} \leq s^{-1/2}||h_{T_{j-1}}||_{l_1}$, and now we just add these up for $j \geq 2$. \\
Also, $T_0^c = T_1 \bigcup T_2 \bigcup \dots$. Thus, adding all the elements of $h_{T_0^c}$ is nothing but adding all the elements from $h_{T_1}, h_{T_2},\dots$. Thus, $||h_{T_1}||_{l_1}+||h_{T_2}||_{l_1}+\dots = ||h_{T_0^c}||_{l_1}$. Combining the results, we get
\[
||h_{T_j}||_{l_2}\leq s^{-1/2}(||h_{T_1}||_{l_1}+||h_{T_2}||_{l_1}+\dots) \leq s^{-1/2}||h_{T_0^c}||_{l_1}
\]

\subsection*{Step 4}
$||\sum_{j \geq 2} h_{T_j}||_{l_2} \leq \sum_{j \geq 2} ||h_{T_j}||_{l_2}$ is true, directly by applying Triangle inequality, as in Step 1. This can be generalised to any number of vectors by sequentially applying it on two vectors at a time. 

\subsection*{Step 5}
$\sum_{j \geq 2}||h_{T_j}||_{l_2} \leq s^{-1/2}||h_{T_0^c}||_{l_1}$ comes directly from Step 3. Thus, we get 
\[
||\sum_{j \geq 2} h_{T_j}||_{l_2} \leq \sum_{j \geq 2} ||h_{T_j}||_{l_2} \leq s^{-1/2}||h_{T_0^c}||_{l_1}
\]

\subsection*{Step 6}
By triangle inequality, we know $|a+b| \geq |a|-|b|$. Thus, $\sum_{i\in T_0} |x_i+h_i| \geq \sum_{i \in T_0} |x_i| - \sum_{i \in T_0} |h_i| = ||x_{T_0}||_{l_1} - ||h_{T_0}||_{l_1}$.\\
Similarly, $\sum_{i\in T_0^c} |x_i+h_i| \geq \sum_{i \in T_0^c} |h_i| - \sum_{i \in T_0^c} |x_i| = ||h_{T_0^c}||_{l_1} - ||x_{T_0^c}||_{l_1}$.\\
Adding the two inequalities, we get 
\[
\sum_{i\in T_0} |x_i+h_i| + \sum_{i\in T_0^c} |x_i+h_i| \geq ||x_{T_0}||_{l_1} - ||h_{T_0}||_{l_1} + ||h_{T_0^c}||_{l_1} - ||x_{T_0^c}||_{l_1}
\]

\subsection*{Step 7}
We have $||x||_{l_1} = ||x_{T_0}||_{l_1}+||x_{T_0^c}||_{l_1}\geq ||x_{T_0}||_{l_1} - ||h_{T_0}||_{l_1} + ||h_{T_0^c}||_{l_1} - ||x_{T_0^c}||_{l_1}$ from the previous step. \\
Cancelling $||x_{T_0}||_{l_1}$ from both sides, and rearranging the inequality, we get 
\[
||h_{T_0^c}||_{l_1} \leq ||h_{T_0}||_{l_1} + 2||x_{T_0^c}||_{l_1}
\]

\subsection*{Step 8}
We can write $h = h_{T_0} + h_{T_1} + h_{(T_0 \cup T_1)^c}$. We know that $\sum_{j \geq 2}||h_{T_j}||_{l_2} \leq s^{-1/2}||h_{T_{0}^c}||_{l_1}$.\\
We also know that $||h_{T_0^c}||_{l_1} \leq ||h_{T_0}||_{l_1} + 2||x_{T_0^c}||_{l_1}$. \\
Thus, we get $\sum_{j \geq 2}||h_{T_j}||_{l_2} \leq  s^{-1/2}(||h_{T_0}||_{l_1} + 2||x_{T_0^c}||_{l_1})$.\\
Also, $||h_{(T_0\cup T_1)^c}||_{l_2} = ||\sum_{j\geq2}h_{T_j}||_{l_2} \leq  \sum_{j\geq2}(||h_{T_j}||_{l_2}) \leq s^{-1/2}(||h_{T_0}||_{l_1} + 2||x_{T_0^c}||_{l_1})$ .\\
Note that the 2nd step is by Triangle Inequality.\\
Applying Cauchy-Schwarz on $h_{T_0}$ and a vector with ones at the positions at which $h_{T_0}$ is non-zero, we get $||h_{T_0}||_{l_1}^2 \leq s||h_{T_0}||_{l_2}^2$. \\
Thus, $s^{-1/2}||h_{T_0}||_{l_1} \leq ||h_{T_0}||_{l_2}$. Hence, using this, we get 
\[
||h_{(T_0\cup T_1)^c}||_{l_2} \leq ||h_{T_0}||_{l_2} + 2s^{-1/2}||x_{T_0^c}||_{l_1} = ||h_{T_0}||_{l_2} + 2s^{-1/2}||x-x_s||_{l_1}
\]
which is what is required.

\subsection*{Step 9}
The first inequality is a direct application of Cauchy Schwarz on $\Phi h_{T_0 \cup T_1}$ and $\Phi h$.
\\
From the problem formualtion, we know that $x^*$ is the solution to $\min_x ||x||_{l_1}$ subject to $||\Phi x - y||_{l_2} \leq \varepsilon$.\\
From step 1, $||\Phi h||_{l_2} = ||\Phi (x^*-x)||_{l_2} \leq 2\varepsilon$.\\
From RIP, we know that $\Phi$ satisfies the RIP with constant $\delta_{2s}$. Thus,\\ $||\Phi h_{T_0 \cup T_1}||_{l_2} \leq \sqrt{1+\delta_{2s}} ||h_{T_0 \cup T_1}||_{l_2}$. Multiplying the inequalities, we get $||\Phi h||_{l_2}||\Phi h_{T_0 \cup T_1}||_{l_2} \leq 2\varepsilon\sqrt{1+\delta_{2s}} ||h_{T_0 \cup T_1}||_{l_2}$, which is the required result.

\subsection*{Step 10}
This is a direct application of Lemma 2.1, where $x$ is $h_{T_0}$ and $x'$ is $h_{T_j}$. The proof involves using the parallelogram identity of the sum of two vectors.\\
Thus, we have $|<\Phi h_{T_0},\Phi h_{T_1}>| \leq \delta_{2s} ||h_{T_0}||_{l_2}||h_{T_j}||_{l_2}$

\subsection*{Step 11}
Since $T_0$ and $T_1$ are disjoint, $||h_{T_0 \cup T_1 }||_{l_2}^2 = ||h_{T_0}||_{l_2}^2+||h_{T_1}||_{l_2}^2 $.\\
Also, $(a-b)^2 \geq 0 \implies2(a^2+b^2) \geq (a+b)^2$. Substituting $a$ as $h_{T_0}$ and $b$ as $h_{T_1}$, and taking square root, we get
\[
2||h_{T_0 \cup T_1}||_{l_2}^2 \geq (||h_{T_0}||_{l_2}+||h_{T_1}||_{l_2})^2
\]
Taking square root on both sides gives the desired result.

\subsection*{Step 12 to Step 16 are Handwritten}
\end{solution}


\end{document}
