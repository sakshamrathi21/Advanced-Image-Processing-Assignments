\documentclass[a4paper,12pt]{article}
\usepackage{xcolor}
\usepackage{amsmath,amsfonts,amssymb}
\usepackage{geometry}
\usepackage{fancyhdr}
\usepackage{graphicx}
\usepackage{titlesec}
\usepackage{tikz}
\usepackage{booktabs}
\usepackage{array}
\usetikzlibrary{shadows}
\usepackage{tcolorbox}
\usepackage{float}
\usepackage{lipsum}
\usepackage{mdframed}
\usepackage{pagecolor}
\usepackage{mathpazo}   % Palatino font (serif)
\usepackage{microtype}  % Better typography

% Page background color
\pagecolor{gray!10!white}

% Geometry settings
\geometry{margin=0.5in}
\pagestyle{fancy}
\fancyhf{}

% Fancy header and footer
\fancyhead[C]{\textbf{\color{blue!80}CS754 Assignment-1}}
% \fancyhead[R]{\color{blue!80}Saksham Rathi}
\fancyfoot[C]{\thepage}

% Custom Section Color and Format with Sans-serif font
\titleformat{\section}
{\sffamily\color{purple!90!black}\normalfont\Large\bfseries}
{\thesection}{1em}{}

% Custom subsection format
\titleformat{\subsection}
{\sffamily\color{cyan!80!black}\normalfont\large\bfseries}
{\thesubsection}{1em}{}

% Stylish Title with TikZ (Enhanced with gradient)
\newcommand{\cooltitle}[1]{%
  \begin{tikzpicture}
    \node[fill=blue!20,rounded corners=10pt,inner sep=12pt, drop shadow, top color=blue!50, bottom color=blue!30] (box)
    {\Huge \bfseries \color{black} #1};
  \end{tikzpicture}
}
\usepackage{float} % Add this package

\newenvironment{solution}[2][]{%
    \begin{mdframed}[linecolor=blue!70!black, linewidth=2pt, roundcorner=10pt, backgroundcolor=yellow!10!white, skipabove=12pt, skipbelow=12pt]%
        \textbf{\large #2}
        \par\noindent\rule{\textwidth}{0.4pt}
}{
    \end{mdframed}
}

% Document title
\title{\cooltitle{CS754 Assignment-2}}
\author{{\bf Saksham Rathi, Ekansh Ravi Shankar, Kshitij Vaidya}}
\date{}

\begin{document}
\maketitle
\textbf{Declaration:} The work submitted is our own, and
we have adhered to the principles of academic honesty while completing and submitting this work. We have not referred to any unauthorized sources, and we have not used generative AI tools for the work submitted here.

\section*{Question 4}

\begin{solution}{Part (a)}
  We are given a matrix $A$ and a vector $v$ such that $Av = 0$. The condition $||v_S||_1 \leq ||v_{\bar{S}}||_1$ is true for set $S$ that contains the indices of the $s$ largest absolute value entries of $v$. 

  We will prove that this is true for any other set $S$ such that $|S| \leq s$. 

  Let us assume to the contrary that there exists a set $R$ such that $|R| = k \leq s$ and $||v_R||_1 > ||v_{\bar{R}}||_1$.

  By definition, $S$ contains the indices of the $s$ largest absolute value entries of $v$, meaning that $\sum_{i\in S}|v_i|$ is maximized over all subsets of size $s$. Moreover, the set $R$ has a cardinality of $k \leq s$, therefore, $\sum_{i\in R}|v_i| \leq \sum_{i\in S}|v_i|$. 

  Combining these three inequalities, we get:
\begin{equation}
  ||v_{\bar{S}}||_1 \geq ||v_S||_1 \geq ||v_R||_1 > ||v_{\bar{R}}||_1
\end{equation}

From above, we get $||v_{\bar{S}}||_1 + ||v_S||_1 > ||v_R||_1 + ||v_{\bar{R}}||_1$. However, we also have $||v_{\bar{S}}||_1 + ||v_S||_1 = ||v||_1 = ||v_R||_1 + ||v_{\bar{R}}||_1$. Therefore, we have a contradiction, and the assumption that there exists a set $R$ such that $|R| = k \leq s$ and $||v_R||_1 > ||v_{\bar{R}}||_1$ is false.

\end{solution}

\begin{solution}{Part (b)}
  We define $\sigma_{s,1} = inf_{||w_0|| \leq s}||v-w||_1$. We need to minimize the $l_1$ norm of $v-w$ over all vectors $w$ that have at most $s$ non-zero entries. For this, we can choose $w$ to be the vector $v$ with all but the $s$ largest absolute value entries set to zero (a greedy choice). Now, the value of $\sigma_{s,1}$ is equal to the absolute sum of $n-s$ entries of $v$. (This is somewhat similar to what we had taken in the previous part.)

  We need to prove that MNSP implies that $||v||_1 \leq 2\sigma_{s,1}$. From MNSP, we have that if $v \in nullspace(A) - \{0\}$, then $||v_S||_1 \leq ||v_{\bar{S}}||_1$. We can write $||v||_1 = ||v_S||_1 + ||v_{\bar{S}}||_1$ (this is true for any set $S$, as the indices of $v$ can be  exhaustively partitioned into $S$ and $\bar{S}$). So, this property also holds for the set $S$ which contains $s$ largest absolute value entries of $v$. 

  Now, we have $\sigma_{s,1} = ||v_{\bar{S}}||_1$ (for such a set $S$). Also, we have:

  \[
    ||v_S||_1 \leq ||v_{\bar{S}}||_1 = \sigma_{s,1}
  \]

  The above is true for a matrix which satisfy MNSP. Therefore:

  \[
    ||v||_1 = ||v_S||_1 + ||v_{\bar{S}}||_1 \leq 2\sigma_{s,1}
  \]
\end{solution}

\begin{solution}{Part (c)}
  Now we need to prove:
  Given a matrix $A \in \mathbb{R}^{m\times n}$, any s-sparse vector $x \in \mathbb{R}$ is a unique solution to the $P1$ problem with the constraint $y = Ax$ iff $A$ satisfies the MNSP of order $s$. There are two parts of this proof (for both directions).

  \subsection*{$A$ satisfies MNSP $\implies$ $x$ is a unique solution to $P1$}
  The P1 problem is given by:
  \[
    \min_{z}||z||_1 \quad \text{subject to} \quad Az = y
  \]
  Assume to the contrary that there exists another $s$-sparse vector $x'$ such that $Ax' = y$ and $$||x||_1 = ||x'||_1$$ (else we can pick one of them as the unique solution). 

  Since, both $x$ and $x'$ are $s-sparse$, let $S$ and $S'$ be the support sets of $x$ and $x'$ respectively. We have $||x||_1 = ||x'||_1 \implies ||x_S||_1 = ||x'_{S'}||_1$ (as other enties are 0). Let us take $v = x - x'$. We have $Av = 0$ (as $Ax = Ax'$). Since, $x$ and $x'$ are $s$-sparse, $v$ is $2s$-sparse. 

  From the previous part (since $A$ satisfies MNSP of order $s$), we have $||v_R||_1 \leq ||v_{\bar{R}}||_1$, for all subsets $R$ of size $|R| \leq s$. Let us take $R$ to be the indices corresponding to the $s$ largest absolute value entries of $v$. Now, since $v$ is $2s-sparse$, $||v_{\bar{R}}||_1$ is the sum of remaining $t \leq s$ non-zero entries of $v$ (after we have chosen the $s$ largest absolute value entries).
  
  Now, since we have already chosen $R$ to contain the $s$ largest absolute value entries of $v$, we have $||v_R||_1 \geq ||v_{\bar{R}}||_1$ (because $\bar{R}$ can't contain values higher than $R$). This leaves us with the case that $||v_R||_1 = ||v_{\bar{R}}||_1$. This is essentially the case, where $v$ contains exactly $2s$ non-zero entries and all of them are equal (as the sum of largest $s$ is equal to the sum of smallest $s$ values). 
\end{solution}


\end{document}
